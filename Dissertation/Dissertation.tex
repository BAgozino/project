% Draft #1
\documentclass[12pt,twoside,notitlepage]{report}
\usepackage{a4wide}
\usepackage{hyperref}
\usepackage{graphicx}
\usepackage{epstopdf}

\parindent 0pt
\parskip 6pt

\begin{document}

%%%%%%%%%%%%%%%%%%%%%%%%%%%%%%%%%%%%%%%%%%%%%%%%%%%%%%%%%%%%%%%%%%%%%%%%
% Title


\pagestyle{empty}

\hfill{\LARGE \bf Biko Agozino}

\vspace*{60mm}
\begin{center}
\Huge
{\bf Inferring sequence specification from Drum Rhythms} \\
\vspace*{5mm}
Computer Science Tripos - Part II \\
\vspace*{5mm}
St John's College \\
\vspace*{5mm}
\today  % today's date
\end{center}

\cleardoublepage

%%%%%%%%%%%%%%%%%%%%%%%%%%%%%%%%%%%%%%%%%%%%%%%%%%%%%%%%%%%%%%%%%%%%%%%%%%%%%%
% Proforma, table of contents and list of figures

\setcounter{page}{1}
\pagenumbering{roman}
\pagestyle{plain}

\chapter*{Proforma}

{\large
\begin{tabular}{ll}
Name:               & \bf Biko Agozino                       \\
College:            & \bf St John's College                     \\
Project Title:      & \bf  \\
Examination:        & \bf Computer Science Tripos - Part II, June 2016       \\
Word Count:         & \bf TODO \\
Project Originator: & Dr A.~Blackwell \& Dr S.~Aaron                \\
Supervisor:         & Mr I.~Herman                    \\ 
\end{tabular}
}

\stepcounter{footnote}


\section*{Original Aims of the Project}

To build a system that is able to infer the intended sequence of
pattern of strokes on a drum kit from those that are captured. The system then aims at matching this sequence against possible popular rock songs that were originally performed before the year 2002 regardless of the variations in pressure and timing of each individual stroke and the performance as a whole. The project aims to detail the possibility of using distance metrics on short drum performances as a basis for imitation based querying of drum notation.

\section*{Work Completed}
The system has been built to infer and extract a candidate sample pattern from MIDI performance data by examining for repeated sequences in a repeated bar performance. This sample pattern is then compared with the database and returns the closest match within some bound. Multiple distance metrics have been implemented to assess the advantages and disadvantages of each in this domain.

A parser has been built to parse informal ASCII Drum Tablature that has been successful in parsing (TODO percent) of the notation to an appropriate format for use in the database that the system queries against.


\section*{Special Difficulties}

None
 
\newpage
\section*{Declaration}

I, Biko Agozino of St John's College, being a candidate for Part II of the Computer Science Tripos, hereby declare that this dissertation and the work described in it are my own work, unaided except as may be specified below, and that the dissertation does not contain material that has already been used to any substantial extent for a comparable purpose.

\bigskip
\leftline{Signed [signature]}

\medskip
\leftline{Date [date]}

\cleardoublepage

\tableofcontents


\newpage


%%%%%%%%%%%%%%%%%%%%%%%%%%%%%%%%%%%%%%%%%%%%%%%%%%%%%%%%%%%%%%%%%%%%%%%
% now for the chapters

\cleardoublepage        % just to make sure before the page numbering
                        % is changed

\setcounter{page}{1}
\pagenumbering{arabic}
\pagestyle{headings}

\chapter{Introduction}

	This dissertation describes the implementation and evaluation of an query-by-example method for searching collaborative databases of drum rhythms, transcribed by users, found in pre-2002 contemporary music.
	
	\section{Motivation}
	The relationship between computers and music has been apparent since the 1950s when Trevor Pearcey and Maston Beard pioneered what was likely the first computer capable perform music, CSIRAC\footnotemark \footnotetext{Council for Scientific and Industrial Research Automatic Computer}, which was able to broadcast music programmed on punched-paper data tape in a similar notation to standard musical notation\cite{CSIRAC}. At the time it took specialised, multimillion-dollar, computers hours or even days to generate a few minutes of simple tunes\cite{Mathews1963}.
	
	Since the early days of Computer Music research, computers have become both more accessible and more powerful. This has allowed computers to be more frequently used in the composition of music by both amateur and professional musicians. This coupled with the onset of MIDI\footnotemark \footnotetext{Musical Instrument Digital Interface: Developed in the early 1980s\cite{Midi1995}} that coincided samplers and digital synthesisers and samplers - which opened possibilities to hobbyists to have studios in their homes and develop their own sounds. Many of the popular genres we hear today were spawned as a result of this musical revolution.
	
	More recently a significant amount of research has been done into QBE\footnotemark \footnotetext{Query-by-Example} systems as they pertain to music. Most notably Shazam\cite{Shazam}, which is a commercial service that aims at music recognition by using audio samples, that may be distorted, in order to query a database of over 3 million tracks.
	
	Due to this relation between computers and music, along with the commercial use and success of many musical platforms, it is clear that further research into computer music is a worthwhile pursuit. Further to this, the prevalence of using technology to produce music has shown that continued development of platforms that make it easier for artists to make music is important.
	
	In contemporary musical ensembles the drummer usually serves the vital role of providing the tempo, pace, and rhythm of the performance. Due to this the drummer can often define the entire song as all other musicians typically use the drummer as the context in which to frame their timing. Therefore, many genres, can be defined by their use of drums.
	
	When composing or practising a new song it is useful, from a drummers perspective, to know how similar rhythms were used in the wider context of the song. This has been the primary motivation behind developing a QBE system that relies only on the drum rhythms present in a song. 
	
	The main use case that has driven design is as follows. A drummer has a drum rhythm in mind, perhaps they have heard it in a song in the past and wants to either see where else it has been used in the past, or perhaps they have designed it themselves and want to see where it has been performed. They play the rhythm on a drumkit as they would usually, which then queries the database and returns all of the similar rhythms.
	
	My goal in developing the system was to build a natural drum rhythm recognition system that:
	\begin{itemize}
		\item{Returns both the inferred rhythm and all rhythms that are similar found in the database}
		\item{Does not require an extensive digital user interface;}
		\item{Extracts the rhythm from repeated play;}
		\item{Be trivially extended to work in real time}
	\end{itemize}	

	\section{Challenges}
	With any information retrieval system it is important to consider the database available. At the start of this project there was no easily computer-readable database of drum rhythms. Therefore a significant proportion of the project was spent on how to best compile this resource.
	
	Furthermore, as with any imitation-based QBE system it can be difficult to extract relevant features when the user is unable to provide a good example. In this case, each drummer has their own nuances to their performance that manifest in the form of a slower/faster tempo or softer/harder beats. This can make it difficult for users when attempting to reproduce the performance, as the intended patterns have been subject to their own personal transformations.
	
	
	\section{\label{sec:SummaryOfRelatedWork}Summary of related work}
	
	Machine learning approaches

	Imitation based querying

	Sequence matching

\cleardoublepage

\chapter{Preparation}

	With any project of a non-trivial size it is important to spend a good amount of time carefully planning and organising tasks. In the case of this project it proved particularly useful when it became clear that the implementation of the system had to significantly change, from the one that was originally proposed, when the available resources and preferred system features were taken into account.
	
	This chapter outlines the investigations that were undertaken prior to the projects implementation, along with detailed arguments for design decisions. I begin by discussing the task of building an adequate database, followed by a discussion on the possible querying methods. Finally, I end by discussing the tools that will be used to implement the project.
	
	\section{Database collection}
	
	Clearly, the success of the project heavily relies upon whether an adequate database can be built. Therefore, it is a task that I decided to tackle first so as to provide the base for the future work. Here I will discuss the possible data sources that were considered along with a comparison and argument for the final choice.
	
		\subsection{Beat tracking techniques}
		
		\begin{figure}[h]
			\centerline{\includegraphics[scale=0.5]{figures/PLACEHOLDER.eps}}
			\caption{\label{beatTracking} Shown here is the aim of beat tracking audio signals.}
\end{figure}

		Beat tracking, as defined by Goto\cite{Goto2001}, is the process of inferring the hierarchical beat structure from musical audio signals. The idea here, depicted in figure \ref{beatTracking}, is that real-time audio from popular music songs can be used as a source of data in order to extract features about the song. 
		
		The advantage of this is that there is that the amount of data available is effectively limitless and, if perfect beat tracking were achievable, the rhythm extracted will be a perfect representation of the underlying rhythm. Additionally this approach may allow further features, apart from the beat timings, to contribute to the database - allowing for a richer searching experience.
		
		Perfect beat tracking, however, is a very challenging field. Recent attempts at solving the problem (\cite{Ellis2007} \cite{EllisPoliner2007} \cite{DaviesPlumbley2007}) have reported efficient algorithms but with only an accuracy of around 60\%.
		\subsection{Feature extraction from notation}
		There are two distinct techniques that can be implemented in order to extract features of drum rhythms from rhythm notation that I will outline here. I will discuss each in turn before returning to the comparison between methods in \ref{subsec:ChoiceOfDataSource}
			\subsubsection{Optical Music Recognition}
			\begin{figure}[h]
			\centerline{\includegraphics[scale=0.5]{figures/PLACEHOLDER.eps}}
			\caption{\label{PercussionNotation} Shown here is a standard percussion notation bar}
\end{figure}
		Musical notation has been a practice in many cultures since as long as we have record\cite{Scelta}. As such the notation has been refined over thousands of years, and what we now think of as 'standard notation' has been firmly established since the 18th century\cite{Scelta}.
		
		Percussion notation, a specific music notation that pertains to percussion instruments, is much less standardised and only (relatively) recently has there been a push for a standardised practice\cite{Weinberg1994}. 
		
		There has been a significant research\cite{Johansen2009}\cite{BainbridgeBell2001} into Music OCR\footnotemark \footnotetext{Music Optical Character Recognition, sometimes referred to as Optical Music Recognition}, which involves using computer vision techniques in order to allow computers to read sheet music. Applying these techniques to percussion notation can enable extraction of feature sets that can be queried.
		
			\subsubsection{ASCII Drum Tablature Parsing}
						\begin{figure}[h]
			\centerline{\includegraphics[scale=0.5]{figures/PLACEHOLDER.eps}}
			\caption{\label{DrumTablature} Shown here is an ASCII drum tablature bar}
\end{figure}
				In drum tablature, each line corresponds to a single component of the drumkit, as shown in figure \ref{DrumTablature}. This contrasts with standard notation, where the height of each note refers to the pitch of the note. Enthusiast transcribers have taken to the ASCII character-encoding scheme in order to share their work over collaborative databases (can I reference http://drumbum.com/drumtabs/ ??).
				
				This ASCII format, while intend to be read by people, may prove to be a great candidate for a general tablature parser. In fact, this problem has been attempted to be solved in a similar case of guitar tablature\cite{Knowles2013}. However, this solution does not solve it in the general case where the tablature may contain comments to help the human reader. My solution will have to account for this as the collaborative database will not be perfectly void of errors.
		\subsection{\label{subsec:ChoiceOfDataSource}Choice of data source}
		Both beat tracking and music OCR are very challenging fields, each large enough to justify their independent research. ASCII Drum Tablature is very feasible in the time frame and given the large amount of data available in this format it should be feasible to collate an adequate database.
		
		It is important, however, to keep in mind the disadvantages of selecting this data source:
		\begin{enumerate}
			\item{As it is a collaborative database, we can not easily verify the accuracy of the transcriptions}
			\item{As there is no standardised format a parser can only be tailored towards a specific practice}
		\end{enumerate}
		
		
		
	\section{Machine learning vs Information retrieval}
	This may make more sense in the querying section
	\section{Querying}
	Now that the data source has been decided, it is important to discuss how the querying system will work. I will start by outlining methods for inferring the bar from a sequence of beats on the drum. I will then discuss the possible ways of using this sequence to query the database.
		\subsection{Suffix trees}
		A suffix tree\cite{Weiner1973} is a string data structure that is useful for many operations on sequences of data. We are mainly interested in it for its ability to find repeated structures in linear time\cite{Gusfield1999}.					\begin{figure}[h]
			\centerline{\includegraphics[scale=0.5]{figures/PLACEHOLDER.eps}}
			\caption{\label{SuffixTree} Suffix tree of the String "BANANA\$"}
\end{figure}
		
			\subsubsection{Ukonnens algorithm}
			Ukonnens algorithm\cite{Ukkonen1995} is an on-line solution to suffix tree construction. This is important for the implementation as for the system to work in real time the suffix tree building must efficiently work in real time. As shown in \ref{subsubsec:Ukonnens} the algorithm works in both linear time and space if run off-line which is optimal.
		\subsection{Matching metrics}
		When we have a candidate sequence to query the database we will need to consider how to evaluate which rhythms are similar to be relevant to the user. I will discuss two distinct methods and explain my choice towards one of them 
			\subsubsection{String metrics}
			String metrics measure the reverse similarity, or distance, between two strings that cab 
			
			\subsubsection{Matching}
			RS Bird algo
	\section{Tools Used}
		\subsection{Midi}
		describe midi
		\subsection{Jack}
		describe jack
		\subsection{Misc}
		
	\section{Summary}
	This chapter discussed the initial research I did into the relevant areas of Computer Science and Computer Music. This chapter then discusses the advantages and disadvantages of each possible method, ending with a conclusion as to why I decided to make certain design decisions in each section. Finally this chapter discusses the tools used for the implementation of the project
\cleardoublepage
\chapter{Implementation}
	\section{Overview of system}
	
	-parser
	-midi input
	-bar extraction
	-matching
	\section{Parser}
	As outlined in the (todo: refenence section where I decided to use ASCII drum tablature) the source for the database will in the format of ASCII drum tablature, which is intended to be human-readable, and so parsing it is not a trivial problem. 
	
	As shown in figure \ref{exampleIdealParse} the parser takes an ASCII file containing specifications of the rhythms found in the song and delivers each bar-long rhythm, saved seperately, 
	
	
	-important to remove all erronous sequences

	-define the "Ideal" format of Tablature

	-7.8mb of parsed data at 135bytes each approx 60000 rhythms

	-7.8MB vs 11.9MB
	
		\subsection{Overview of Parser}
\begin{figure}[h]
			\centerline{\includegraphics[scale=0.5]{figures/PLACEHOLDER.eps}}
			\caption{\label{exampleIdealParse} Here is an example of how the system will translate and format the drum tablature for pattern matching in the database}
\end{figure}

		\subsection{Lexical Analysis}
		In the parser, the lexical analysis involves tokenising the text to 
		
		Token all the text with regular expressions
			\subsubsection{Comment removal}
			Step through some steps of the algorithm in a place wherer comments will be removed
		\subsection{Information lost}
		Outline cases where we have lost data
	\section{Bar Extraction}
	We input into the system as a midi file with repeated play and use suffix trees to extract the beat
		\subsection{\label{subsec:SuffixTree}Suffix Tree's}
		Define suffix tree's and explain why useful
			\subsubsection{\label{subsubsec:Ukonnens}Ukonnens Algorithm}
			Explain why we need an online solution from a usability standpoint
			Overview of algorithm
			\subsubsection{Supermaximal repeats}
			Definition
			How to get them from suffix tree (Gusfield)
			
		\subsection{Bar inferrence}
		Use Supermaximal repeat as a reference.
		Longest Supermaximal repeat is the bar
			\subsubsection{Quantisation}
			split bar into 16ths
	\section{Distance metrics}
	Definition of Distance metrics
		\subsection{2D Edit distance}
		
		\subsection{Hamming Distance}
		
		\subsubsection{Cyclic extensions}
			
	\section{Summary}
	

\cleardoublepage
\chapter{Evaluation}
	\section{Comparison with original aims}
	\section{Testing}
		\subsection{Dataset collection}
	\section{Performance Analysis}
	\section{Summary}
	
\cleardoublepage
\chapter{Conclusion}
	\section{Lessons learnt}
	\section{Future work}

\cleardoublepage

%%%%%%%%%%%%%%%%%%%%%%%%%%%%%%%%%%%%%%%%%%%%%%%%%%%%%%%%%%%%%%%%%%%%%
% the bibliography

\addcontentsline{toc}{chapter}{Bibliography}
\bibliography{refs}
\begin{thebibliography}{99}
	\bibitem{Shazam}
	Wang, Avery.
	\emph{The Shazam music recognition service},
	Communications of the ACM,
	2006.
	\bibitem{Goto2001}
	Goto, Masataka.
	\emph{An audio-based real-time beat tracking system for music with or without drum-sounds},
	Journal of New Music Research,
	2001.
	\bibitem{Ellis2007}
	Ellis, Daniel PW.
	\emph{Beat tracking by dynamic programming},
	Journal of New Music Reasearch,
	2007.
	\bibitem{EllisPoliner2007}
	Ellis, Daniel PW; Poliner, Graham E.
	\emph{Identifying cover songs' with chroma features and dynamic programming beat tracking}
	Acoustics, Speech and Signal Processing,
	2007.
	\bibitem{DaviesPlumbley2007}
	Davies, Matthew EP; Plumbley, Mark D.
	\emph{Context-dependent beat tracking of musical audio}
	Audio, Speech, and Language Processing,
	2007.
	\bibitem{Mathews1963}
	Mathews, Max.
	\emph{The Digital Computer as a Musical Instrument},
	Science,
	1963.
	\bibitem{Midi1995}
	Rothstein, Joseph.
	\emph{Midi: A comprehensive introduction}
	AR Editions,
	1995.
	\bibitem{BainbridgeBell2001}
	Bainbridge, David; Bell, Tim.
	\emph{The challenge of optical music recognition}
	Computers and Humanities,
	2001
	\bibitem{ComputerMusic}
	Dodge, Charles; Thomas A. Jerse.
	\emph{Computer music: synthesis, composition and performance},
	Macmillan Library Reference, 
	1997.
	\bibitem{Scelta}
	Scelta, Gabriella F.
	\emph{The History and Evolution of the Musical Symbol}.
	\bibitem{Weinberg1994}
	Weinberg, Norman.
	\emph{Guidelines for Drumset Notation},
	Percussive Notes,
	1994.
	\bibitem{Johansen2016}
	Johansen, Linn S.
	\emph{Optical music recognition},
	2009.	
	\bibitem{Knowles2013}
	Knowles, Evan.
	\emph{Parsing Guitar Tab},
	http://knowles.co.za/parsing-guitar-tab/,
	2013.
	\bibitem{CSIRAC}
	Doornbusch, Paul.
	\emph{Computer sound synthesis in 1951: the music of CSIRAC},
	Computer Music Journal,
	2004.
	\bibitem{Weiner1973}
	Weiner, Peter.
	\emph{Linear Pattern Matching Algorithms},
	The Rand Corporation,
	1973.
	\bibitem{Gusfield1999}
	Gusfield, Dan.
	\emph{Algorithms on Strings, Trees and Sequences: Computer Science and Computational Biology}
	Cambridge University Press,
	1999.
	\bibitem{Ukkonen1995}
	Ukkonen, Esko.
	\emph{On-line construction of suffix trees}
	University of Helsinki,
	1995.


\end{thebibliography}
\cleardoublepage

%%%%%%%%%%%%%%%%%%%%%%%%%%%%%%%%%%%%%%%%%%%%%%%%%%%%%%%%%%%%%%%%%%%%%
% the appendices
\appendix

\end{document}
